% Options for packages loaded elsewhere
\PassOptionsToPackage{unicode}{hyperref}
\PassOptionsToPackage{hyphens}{url}
%
\documentclass[
  11pt,
]{article}
\usepackage{lmodern}
\usepackage{setspace}
\usepackage{amsmath}
\usepackage{ifxetex,ifluatex}
\ifnum 0\ifxetex 1\fi\ifluatex 1\fi=0 % if pdftex
  \usepackage[T1]{fontenc}
  \usepackage[utf8]{inputenc}
  \usepackage{textcomp} % provide euro and other symbols
  \usepackage{amssymb}
\else % if luatex or xetex
  \usepackage{unicode-math}
  \defaultfontfeatures{Scale=MatchLowercase}
  \defaultfontfeatures[\rmfamily]{Ligatures=TeX,Scale=1}
\fi
% Use upquote if available, for straight quotes in verbatim environments
\IfFileExists{upquote.sty}{\usepackage{upquote}}{}
\IfFileExists{microtype.sty}{% use microtype if available
  \usepackage[]{microtype}
  \UseMicrotypeSet[protrusion]{basicmath} % disable protrusion for tt fonts
}{}
\makeatletter
\@ifundefined{KOMAClassName}{% if non-KOMA class
  \IfFileExists{parskip.sty}{%
    \usepackage{parskip}
  }{% else
    \setlength{\parindent}{0pt}
    \setlength{\parskip}{6pt plus 2pt minus 1pt}}
}{% if KOMA class
  \KOMAoptions{parskip=half}}
\makeatother
\usepackage{xcolor}
\IfFileExists{xurl.sty}{\usepackage{xurl}}{} % add URL line breaks if available
\IfFileExists{bookmark.sty}{\usepackage{bookmark}}{\usepackage{hyperref}}
\hypersetup{
  hidelinks,
  pdfcreator={LaTeX via pandoc}}
\urlstyle{same} % disable monospaced font for URLs
\usepackage[margin=1in]{geometry}
\usepackage{color}
\usepackage{fancyvrb}
\newcommand{\VerbBar}{|}
\newcommand{\VERB}{\Verb[commandchars=\\\{\}]}
\DefineVerbatimEnvironment{Highlighting}{Verbatim}{commandchars=\\\{\}}
% Add ',fontsize=\small' for more characters per line
\usepackage{framed}
\definecolor{shadecolor}{RGB}{248,248,248}
\newenvironment{Shaded}{\begin{snugshade}}{\end{snugshade}}
\newcommand{\AlertTok}[1]{\textcolor[rgb]{0.94,0.16,0.16}{#1}}
\newcommand{\AnnotationTok}[1]{\textcolor[rgb]{0.56,0.35,0.01}{\textbf{\textit{#1}}}}
\newcommand{\AttributeTok}[1]{\textcolor[rgb]{0.77,0.63,0.00}{#1}}
\newcommand{\BaseNTok}[1]{\textcolor[rgb]{0.00,0.00,0.81}{#1}}
\newcommand{\BuiltInTok}[1]{#1}
\newcommand{\CharTok}[1]{\textcolor[rgb]{0.31,0.60,0.02}{#1}}
\newcommand{\CommentTok}[1]{\textcolor[rgb]{0.56,0.35,0.01}{\textit{#1}}}
\newcommand{\CommentVarTok}[1]{\textcolor[rgb]{0.56,0.35,0.01}{\textbf{\textit{#1}}}}
\newcommand{\ConstantTok}[1]{\textcolor[rgb]{0.00,0.00,0.00}{#1}}
\newcommand{\ControlFlowTok}[1]{\textcolor[rgb]{0.13,0.29,0.53}{\textbf{#1}}}
\newcommand{\DataTypeTok}[1]{\textcolor[rgb]{0.13,0.29,0.53}{#1}}
\newcommand{\DecValTok}[1]{\textcolor[rgb]{0.00,0.00,0.81}{#1}}
\newcommand{\DocumentationTok}[1]{\textcolor[rgb]{0.56,0.35,0.01}{\textbf{\textit{#1}}}}
\newcommand{\ErrorTok}[1]{\textcolor[rgb]{0.64,0.00,0.00}{\textbf{#1}}}
\newcommand{\ExtensionTok}[1]{#1}
\newcommand{\FloatTok}[1]{\textcolor[rgb]{0.00,0.00,0.81}{#1}}
\newcommand{\FunctionTok}[1]{\textcolor[rgb]{0.00,0.00,0.00}{#1}}
\newcommand{\ImportTok}[1]{#1}
\newcommand{\InformationTok}[1]{\textcolor[rgb]{0.56,0.35,0.01}{\textbf{\textit{#1}}}}
\newcommand{\KeywordTok}[1]{\textcolor[rgb]{0.13,0.29,0.53}{\textbf{#1}}}
\newcommand{\NormalTok}[1]{#1}
\newcommand{\OperatorTok}[1]{\textcolor[rgb]{0.81,0.36,0.00}{\textbf{#1}}}
\newcommand{\OtherTok}[1]{\textcolor[rgb]{0.56,0.35,0.01}{#1}}
\newcommand{\PreprocessorTok}[1]{\textcolor[rgb]{0.56,0.35,0.01}{\textit{#1}}}
\newcommand{\RegionMarkerTok}[1]{#1}
\newcommand{\SpecialCharTok}[1]{\textcolor[rgb]{0.00,0.00,0.00}{#1}}
\newcommand{\SpecialStringTok}[1]{\textcolor[rgb]{0.31,0.60,0.02}{#1}}
\newcommand{\StringTok}[1]{\textcolor[rgb]{0.31,0.60,0.02}{#1}}
\newcommand{\VariableTok}[1]{\textcolor[rgb]{0.00,0.00,0.00}{#1}}
\newcommand{\VerbatimStringTok}[1]{\textcolor[rgb]{0.31,0.60,0.02}{#1}}
\newcommand{\WarningTok}[1]{\textcolor[rgb]{0.56,0.35,0.01}{\textbf{\textit{#1}}}}
\usepackage{longtable,booktabs}
\usepackage{calc} % for calculating minipage widths
% Correct order of tables after \paragraph or \subparagraph
\usepackage{etoolbox}
\makeatletter
\patchcmd\longtable{\par}{\if@noskipsec\mbox{}\fi\par}{}{}
\makeatother
% Allow footnotes in longtable head/foot
\IfFileExists{footnotehyper.sty}{\usepackage{footnotehyper}}{\usepackage{footnote}}
\makesavenoteenv{longtable}
\usepackage{graphicx}
\makeatletter
\def\maxwidth{\ifdim\Gin@nat@width>\linewidth\linewidth\else\Gin@nat@width\fi}
\def\maxheight{\ifdim\Gin@nat@height>\textheight\textheight\else\Gin@nat@height\fi}
\makeatother
% Scale images if necessary, so that they will not overflow the page
% margins by default, and it is still possible to overwrite the defaults
% using explicit options in \includegraphics[width, height, ...]{}
\setkeys{Gin}{width=\maxwidth,height=\maxheight,keepaspectratio}
% Set default figure placement to htbp
\makeatletter
\def\fps@figure{htbp}
\makeatother
\setlength{\emergencystretch}{3em} % prevent overfull lines
\providecommand{\tightlist}{%
  \setlength{\itemsep}{0pt}\setlength{\parskip}{0pt}}
\setcounter{secnumdepth}{5}
% \usepackage{setspace}
% % this is a lorem ipsum generator for adding dummy texts
% \usepackage{lipsum}
% \usepackage{tocloft}
% % to make the first rows bold in tables
% \usepackage{longtable}
% \usepackage{tabu}
% \usepackage{booktabs}
% % this makes list of figures appear in table of contents
% \usepackage[nottoc]{tocbibind}
% 
% % for passing temporary notes
% % \usepackage{todonotes}
% 
% \usepackage{morefloats}
% \usepackage{float}
% 
% % highlighting
% \usepackage{soul}
% 
% % referencing mutliple things with a single command - \cref
% \usepackage{cleveref}
% 
% 
% % this makes dots in table of contents
% \renewcommand{\cftsecleader}{\cftdotfill{\cftdotsep}}
% % to change the title of contents
% % \renewcommand{\contentsname}{Whatever}
% 
% % line numbers for review purposes
% % this package might not be available in default latex installation 
% % get it by 'sudo tlmgr install lineno'
% %\usepackage{lineno}
% %\linenumbers
% 
% % this allows checkmarks in the file
% \usepackage{amssymb}
% \DeclareUnicodeCharacter{2714}{\checkmark}
% 
% % to be able to include latex comments
% \newenvironment{dummy}{}{}




% Universitaet Bern Vorlate LaTex
% \documentclass[12pt,a4paper]{article}


% formatting
% \usepackage[left=4cm, right=3cm, top=3cm, bottom=3cm]{geometry}

% package for nonstandard letters (umlauts etc)
\usepackage{german}
\usepackage[utf8]{inputenc} 

%% Mathematics
\usepackage{amsmath,amsfonts,amssymb}

%% Graphics
\usepackage{graphicx}

%% For nice tables
\usepackage{booktabs}

%% Line spacing
%\renewcommand{\baselinestretch}{1.5}

%% Bibliography
% \usepackage{natbib}

%% to avoid messy linebreaks
\sloppy

\usepackage{csquotes} % deutsche Anfuehrungszeichen
\MakeOuterQuote{"}





\usepackage{flafter}
\ifluatex
  \usepackage{selnolig}  % disable illegal ligatures
\fi

\author{}
\date{\vspace{-2.5em}}

\begin{document}

% Universitaet Bern template
% \begin{document}

\thispagestyle{empty}
\ \vspace{1.0cm}
\begin{center}
{\LARGE
Titel\\[0.1cm]
der Arbeit\\[1cm]
}

{
{\bf Seminararbeit} \\
an der \\[0.5cm]
{\bf Wirtschafts- und Sozialwissenschaftlichen Fakult\"at} \\
{\bf der Universit\"at Bern} \\[0.5cm]
bei \\
Dr. Costanza Naguib\\[8cm]
}


\hspace{1cm}\begin{minipage}[h]{20cm}
\begin{tabbing}
Mastervorlesung: Machine Learning\\
Abgabedatum: 01.06.2020\\[0.5cm]
Ladina Brantschen \hspace{4cm} \= Johannes von Mandach \\
XXXXXX \> XXXXXXX \\
ladina.brantschen@students.unibe.ch \> johannes.vonmandach@students.unibe.ch \\[0.5cm]
Kristina Schüpbach \hspace{4cm} \= Carla Coccia \\
14-116-040 \> XXXXXXX \\
kristina.schuepbach@students.unibe.ch \> carla.coccia@students.unibe.ch \\

\end{tabbing}
\end{minipage}
\end{center}
\newpage



%\begin{abstract}
%Dieses rudiment\"are Template dient als Einstiegshilfe f\"ur die Arbeit mit \LaTeX. Diese Software kann kostenlos aus dem Internet heruntergeladen werden und sollte im Selbststudium angeignet werden; viele Tipps und Tricks finden Sie ebenfalls im Internet.   
%\end{abstract}
%\newpage

% 
% 
% \tableofcontents
% \newpage






{
\setcounter{tocdepth}{2}
\tableofcontents
}
\setstretch{1.5}
\newpage

\hypertarget{einfuxfchrung}{%
\section{Einführung}\label{einfuxfchrung}}

Das ist nur eine ganz kurze Vorlage, der Code funktioniert auch nicht, weil die Daten fehlen. Aber sollte für die paar mühsameren Formatierungen (Tabellen, Grafiken) den relevanten Code drin haben.

\hypertarget{titel}{%
\section{Titel}\label{titel}}

\hypertarget{untertitel}{%
\subsection{Untertitel}\label{untertitel}}

\begin{itemize}
\item
  Unter Help \textgreater{} ``Mardown Quick Reference'' hat es die wichtigsten Formatierungen
\item
  blabla
\end{itemize}

\hypertarget{r-code-einbinden}{%
\subsection{R-Code einbinden:}\label{r-code-einbinden}}

jeder Code-Chunk sollte ein label haben, vereinfacht Debugging

echo = TRUE --\textgreater{} Code \& Output anzeigen

include = TRUE --\textgreater{} nur Output anzeigen (ohne Code)

eval = FALSE --\textgreater{} Code nicht ausführen

\hypertarget{tabelle-erstellen}{%
\subsection{Tabelle erstellen}\label{tabelle-erstellen}}

Am einfachsten ist es, die Tabelle als dataframe zu erstellen, abzuspeichern und dann mit knittr zu formatieren:

\small

\linespread{1}

\begin{Shaded}
\begin{Highlighting}[]
\CommentTok{\# See frequency of all apps}
\NormalTok{tmp }\OtherTok{\textless{}{-}}\NormalTok{ tw }\SpecialCharTok{\%\textgreater{}\%} \FunctionTok{count}\NormalTok{(source, }\AttributeTok{sort =} \ConstantTok{TRUE}\NormalTok{) }\SpecialCharTok{\%\textgreater{}\%} 
  \FunctionTok{head}\NormalTok{(}\AttributeTok{n =} \DecValTok{10}\NormalTok{)}
\NormalTok{knitr}\SpecialCharTok{::}\FunctionTok{kable}\NormalTok{(tmp, }\AttributeTok{format =} \StringTok{"latex"}\NormalTok{, }\AttributeTok{booktabs =} \ConstantTok{TRUE}\NormalTok{, }\AttributeTok{digits =} \DecValTok{2}\NormalTok{, }\AttributeTok{linesep =} \StringTok{""}\NormalTok{,}
             \AttributeTok{caption =} \StringTok{"Die häufigsten Applikationen"}\NormalTok{,}
             \AttributeTok{col.names =} \FunctionTok{c}\NormalTok{(}\StringTok{"Applikation"}\NormalTok{, }\StringTok{"Anzahl Tweets"}\NormalTok{)) }\SpecialCharTok{\%\textgreater{}\%}
\NormalTok{  kableExtra}\SpecialCharTok{::}\FunctionTok{kable\_styling}\NormalTok{(}\AttributeTok{full\_width =} \ConstantTok{FALSE}\NormalTok{, }
                            \AttributeTok{position =} \StringTok{"center"}\NormalTok{,}
                            \CommentTok{\#latex\_options = "hold\_position",}
                            \AttributeTok{latex\_options =} \StringTok{"striped"}\NormalTok{)}
\end{Highlighting}
\end{Shaded}

\linespread{1.5}

\normalsize

Sehr breite Tabellen verkleinern:

\small

\linespread{1}

\begin{Shaded}
\begin{Highlighting}[]
\NormalTok{tmp }\OtherTok{\textless{}{-}} \FunctionTok{select}\NormalTok{(}\FunctionTok{ungroup}\NormalTok{(users), location, res\_long, res\_lat, res\_address, res\_country) }\SpecialCharTok{\%\textgreater{}\%} 
  \FunctionTok{head}\NormalTok{()}

\NormalTok{knitr}\SpecialCharTok{::}\FunctionTok{kable}\NormalTok{(tmp, }\AttributeTok{format =} \StringTok{"latex"}\NormalTok{, }\AttributeTok{booktabs =} \ConstantTok{TRUE}\NormalTok{, }\AttributeTok{digits =} \DecValTok{2}\NormalTok{, }\AttributeTok{linesep =} \StringTok{""}\NormalTok{,}
             \AttributeTok{caption =} \StringTok{"Zuordnung des Herkunftslandes"}\NormalTok{,}
             \AttributeTok{col.names =} \FunctionTok{c}\NormalTok{(}\StringTok{"Standort"}\NormalTok{, }\StringTok{"Logitude"}\NormalTok{,}\StringTok{"Latitude"}\NormalTok{, }\StringTok{"Adresse"}\NormalTok{, }\StringTok{"Land"}\NormalTok{)) }\SpecialCharTok{\%\textgreater{}\%}
\NormalTok{  kableExtra}\SpecialCharTok{::}\FunctionTok{kable\_styling}\NormalTok{(}\AttributeTok{full\_width =} \ConstantTok{TRUE}\NormalTok{, }
                            \AttributeTok{position =} \StringTok{"center"}\NormalTok{,}
                            \AttributeTok{latex\_options =} \FunctionTok{c}\NormalTok{(}\StringTok{"striped"}\NormalTok{, }\StringTok{"hold\_position"}\NormalTok{)}
\NormalTok{                            ) }\SpecialCharTok{\%\textgreater{}\%}
\NormalTok{  kableExtra}\SpecialCharTok{::}\FunctionTok{column\_spec}\NormalTok{(}\DecValTok{1}\NormalTok{, }\AttributeTok{width =} \StringTok{"10em"}\NormalTok{) }\SpecialCharTok{\%\textgreater{}\%}
              \FunctionTok{column\_spec}\NormalTok{(}\DecValTok{2}\SpecialCharTok{:}\DecValTok{3}\NormalTok{, }\AttributeTok{width =} \StringTok{"5em"}\NormalTok{) }\SpecialCharTok{\%\textgreater{}\%}
              \FunctionTok{column\_spec}\NormalTok{(}\DecValTok{4}\NormalTok{, }\AttributeTok{width =} \StringTok{"10em"}\NormalTok{)}
\end{Highlighting}
\end{Shaded}

\linespread{1.5}

\normalsize

\hypertarget{grafiken}{%
\subsection{Grafiken}\label{grafiken}}

Beschriftung von Grafiken und weitere Formatierung (Breite, Ausrichtung) direkt im Header des Codes.

\small

\linespread{1}

\begin{Shaded}
\begin{Highlighting}[]
\CommentTok{\# Plot distribution of probabilities}
\FunctionTok{ggplot}\NormalTok{(pbot, }\FunctionTok{aes}\NormalTok{(prob\_bot)) }\SpecialCharTok{+} 
  \FunctionTok{geom\_density}\NormalTok{(}\AttributeTok{kernel =} \StringTok{"gaussian"}\NormalTok{, }\AttributeTok{color =} \StringTok{"\#C84630"}\NormalTok{, }\AttributeTok{fill =} \StringTok{"\#C84630"}\NormalTok{, }\AttributeTok{alpha =} \FloatTok{0.3}\NormalTok{) }\SpecialCharTok{+}
  \FunctionTok{labs}\NormalTok{(}\AttributeTok{x =} \StringTok{"Wahrscheinlichkeit"}\NormalTok{, }\AttributeTok{y =} \StringTok{"Dichte"}\NormalTok{)  }\SpecialCharTok{+} 
  \FunctionTok{theme\_gray}\NormalTok{() }\SpecialCharTok{+}
  \FunctionTok{theme}\NormalTok{(}
    \AttributeTok{panel.grid.major =} \FunctionTok{element\_line}\NormalTok{(}\AttributeTok{color =} \StringTok{"\#DDDDDA"}\NormalTok{, }\AttributeTok{size =} \FloatTok{0.2}\NormalTok{),}
      \AttributeTok{panel.grid.minor =} \FunctionTok{element\_blank}\NormalTok{(),}
      \AttributeTok{plot.background =} \FunctionTok{element\_rect}\NormalTok{(}\AttributeTok{fill =} \StringTok{"\#f5f5f2"}\NormalTok{, }\AttributeTok{color =} \ConstantTok{NA}\NormalTok{), }
      \AttributeTok{panel.background =} \FunctionTok{element\_rect}\NormalTok{(}\AttributeTok{fill =} \StringTok{"\#f5f5f2"}\NormalTok{, }\AttributeTok{color =} \ConstantTok{NA}\NormalTok{), }
      \AttributeTok{legend.background =} \FunctionTok{element\_rect}\NormalTok{(}\AttributeTok{fill =} \StringTok{"\#f5f5f2"}\NormalTok{, }\AttributeTok{color =} \ConstantTok{NA}\NormalTok{)}
\NormalTok{  )}
\end{Highlighting}
\end{Shaded}

\linespread{1.5}

\normalsize

Mit folgendem Code kann im Text auf bestimmte Grafiken (fig:) bzw. Tabellen (tab:) verlinkt werden:

\ref{fig:botornot-plot}

Manchmal ist es einfacher, die Grösse einer Grafik mit ggsave() einzustellen, sprich die Grafik wird zuerst mit der richtige Grösse exportiert und dann wieder importiert mit include\_graphics()

\small

\linespread{1}

\begin{Shaded}
\begin{Highlighting}[]
\NormalTok{p }\OtherTok{\textless{}{-}} \FunctionTok{ggplot}\NormalTok{() }\SpecialCharTok{+} 
  \CommentTok{\# draw countries}
  \FunctionTok{geom\_polygon}\NormalTok{(}\AttributeTok{data =}\NormalTok{ world, }\FunctionTok{aes}\NormalTok{(}\AttributeTok{x =}\NormalTok{ long, }
                                       \AttributeTok{y =}\NormalTok{ lat, }
                                       \AttributeTok{group =}\NormalTok{ group), }
                 \AttributeTok{fill =} \StringTok{"\#D8D8D2"}\NormalTok{) }\SpecialCharTok{+}
  \FunctionTok{theme\_map}\NormalTok{() }\SpecialCharTok{+}
  \FunctionTok{labs}\NormalTok{(}\AttributeTok{x =} \ConstantTok{NULL}\NormalTok{,}
       \AttributeTok{y =} \ConstantTok{NULL}\NormalTok{) }\SpecialCharTok{+}
  \FunctionTok{theme}\NormalTok{(}\AttributeTok{legend.position =} \StringTok{"none"}\NormalTok{,}
        \AttributeTok{strip.text.y =} \FunctionTok{element\_text}\NormalTok{(}\AttributeTok{size =} \DecValTok{30}\NormalTok{))}
\FunctionTok{ggsave}\NormalTok{(}\StringTok{"figures/worldplot.pdf"}\NormalTok{, p, }\AttributeTok{width =} \DecValTok{16}\NormalTok{, }\AttributeTok{height =} \DecValTok{17}\NormalTok{)}
\end{Highlighting}
\end{Shaded}

\linespread{1.5}

\normalsize

\small

\linespread{1}

\begin{Shaded}
\begin{Highlighting}[]
\NormalTok{knitr}\SpecialCharTok{::}\FunctionTok{include\_graphics}\NormalTok{(}\StringTok{"figures/worldplot.pdf"}\NormalTok{)}
\end{Highlighting}
\end{Shaded}

\linespread{1.5}

\normalsize

\hypertarget{funktionen}{%
\subsection{Funktionen:}\label{funktionen}}

Müssen immer von zwei Dollarzeichen umgeben sein (Funktion auf eigener Zeile):

\[r_{cm} = {1 \over N} \sum_{i \in L} n_{i}r_{i}\]

oder von einem Dollarzeichen (inline-Funktion): \(r_{i}\)

\end{document}
